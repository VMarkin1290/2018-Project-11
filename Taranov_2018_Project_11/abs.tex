\documentclass[12pt,twoside]{article}


\usepackage{jmlda}

\begin{document}
\title{\textsc{•}}
    {Автоматическое определение релевантности параметров нейросети}
\author
    {Таранов $~С.\,К.} % основной список авторов, выводимый в оглавление

\email
    {taranov.sk@phystech.edu}

\abstract
	{В данной работе исследуется выбор оптимальной структуры нейронной сети. Современные модели предпологают значительное число обучаемых параметров, однако предполагается, их число можно снизить с сохранинием достаточной точности. Согласно этой идеи предлагается метод, корректирующий модель в процессе обучения, на основе идеи представления сети в виде графа, рёбра которого являются примитивными функциями, а вершины промежуточными представленими выборки, полученные под действием этих функций. С помощью метода DARTS изучается вопрос выбора оптимальной структуры этого графа, так чтобы модель, ему соответсвующая, в тоже время удовлетворяла требованиям точности для данной задачи. Также проводятся численные эксперименты на выборках данных Boston, MNIST, CIFAR-10.

\bigskip
\textbf{Ключевые слова}: \emph {нейронные сети, оптимизация гиперпараметров, метод DARTS}.

}

\maketitle

\end{document}

